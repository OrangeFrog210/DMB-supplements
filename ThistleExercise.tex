%%
%% For latex -> dvi -> PS or PDF 
%% Works in PCTeX 5 or XeLaTeX in TeXworks 
%%
\documentclass [11pt]{article}
\usepackage {amssymb,amsmath,graphicx,enumitem}

\usepackage[vmargin=1in,hmargin=1in]{geometry}

\setlength{\parskip}{.1in}  
\setlength{\parindent}{0.0in}  

%\renewcommand{\baselinestretch}{2.0} 
\renewcommand{\floatpagefraction}{0.9}
\renewcommand{\topfraction}{0.99}
\renewcommand{\textfraction}{0.1}

\newcommand{\ttt}[1]{\texttt{#1}}
\newcommand{\vp}{\vspace{.20in}}
\newcommand{\bi}{\begin{itemize} \vspace*{-0.1in}}
\newcommand{\be}{\begin{equation}}
\newcommand{\ee}{\end{equation}} 
\newcommand{\ba}{\begin{equation} \begin{aligned}}
\newcommand{\ea}{\end{aligned} \end{equation}}
\newcommand{\bracket}[1]{\left\langle{#1}\right\rangle}
\newcommand{\twovec}[2]{\begin{pmatrix} {#1} \\ {#2} \end{pmatrix}}
\newcommand{\twomat}[4]{\begin{pmatrix} {#1} & {#2} \\ {#3} & {#4}\end{pmatrix}}

\pagestyle{empty}

\newcommand{\tab}{\hspace*{0.5in}}

\sloppy 

\def\X{\mathbf{X}}
\def\A{\mathbf{A}}
\def\B{\mathbf{B}}
\def\C{\mathbf{C}}
\def\D{\mathbf{D}}
%\def\S{\mathbf{S}}
\def\E{\mathbf{E}}
\def\I{\mathbf{I}}
\def\R{\mathbf{R}}
\def\N{\mathbf{N}}
\def\M{\mathbf{M}}
\def\L{\mathbf{L}}

\def\a{\mathbf{a}}
\def\c{\mathbf{c}}
\def\e{\mathbf{e}}
\def\f{\mathbf{f}}
\def\k{\mathbf{k}}
\def\x{\mathbf{x}}
\def\y{\mathbf{y}}
\def\b{\mathbf{b}}
\def\u{\mathbf{u}}
\def\v{\mathbf{v}}
\def\w{\mathbf{w}}
\def\g{\mathbf{g}}
\def\h{\mathbf{h}}
\def\n{\mathbf{n}}
\def\p{\mathbf{p}}
\def\o{\mathbf{o}}

\begin{document}
\begin{center}
\textbf{\large{BIOEE/MATH 3620 Dynamic Models in Biology}} \\
\textbf{Eigenvalue sensitivities and control of thistle populations} \\ 
\end{center}

The goal of this exercise is to let you see for yourself that the population growth rate
$\lambda$ is more sensitive to changes in some matrix entries than others. 
The objects of study are two matrix models created in the script \texttt{ThistleMatrices.R} on Blackboard, 
based on population studies of the agricultural weed \emph{Carduus nutans} (nodding thistle). 

(i) \emph{Carduus nutans} in Australia. The classes are Seeds in the seedbank, and Small, 
Medium, and Large thistles. 
\begin{verbatim} 
 0.4400 6.0000 67.400 183.100
 0.0048 0.4400  3.040   8.300
 0.0000 0.2100  0.042   0.083
 0.0000 0.0089  0.031   0.014
\end{verbatim} 

(ii) \emph{Carduus nutans} in New Zealand; same classes.  
\begin{verbatim}
0.038 8.2500 179.400 503.1
0.180 1.0900  22.200  62.2
0.000 0.0091   0.000   0.0
0.000 0.0056   0.022   0.0
\end{verbatim} 

You will carry out a computational experiment comparing the effects on population growth, and on long-term 
population growth rate $\lambda$, of different perturbations to matrix elements -- and turn in a brief writeup
of your findings. 
\begin{enumerate} 
\item[(a)] Compute the two elasticity matrices. 
\item[(b)] Using the elasticity matrices as your guide, identify (for one of the two matrices) two entries where you 
expect that a relative perturbation to one of the two entries (e.g., a 20\% decrease in the matrix element) will have a 
small impact on $\lambda$, while the same relative perturbation of the other entry will have 
a large impact on $\lambda$. Confirm this by computing what effect the perturbations have on (i) the dominant
eigenvalue $\lambda$, and on (ii) population growth over 25 years. 
\item[(c)] Again using the elasticity matrices as your guide, compare the effects of doing the same  
matrix perturbation in Australia and NZ (e.g., a 20\% decrease in the same matrix element), 
choosing an element where you expect that the effect on $\lambda$ will be relatively small in one location, and 
relatively large in the other. 
\end{enumerate} 

\end{document} 